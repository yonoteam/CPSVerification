\documentclass[english,letterpaper,12pt]{article} %Idioma, papel, 12 puntos, clase.

%------------------------------------------------------------------------------
%	PAQUETES Y CONFIGURACION DEL DOCUMENTO
%------------------------------------------------------------------------------

	% Paquetes
		\usepackage[english, spanish]{babel} 	% Predispone el último idioma de la lista. 
									% Usar \selectlanguage{} para cambiarlo.
		\usepackage[T1]{fontenc} %Permite copypastear cosas con acentos tras crear el pdf.
		\usepackage[utf8]{inputenc} %Permite escribir acentos y ?'s directamente aquí y no por código.
		\usepackage[margin=2.5cm]{geometry} % Afecta los márgenes de manera fácil de entender.
		\usepackage{titlesec} % Allows customization of titles
		\usepackage{setspace} % Spacing between paragraphs
		\usepackage{fancyhdr}%para poner encabezados
	
		\usepackage{graphicx} % Importar imágenes.
		\usepackage{multicol} % Permite cambiar el número de columnas
		\usepackage{booktabs} %Para hacer tablas más fácilmente
		\usepackage{array} % Necesario para usar paquete tabularx
		\usepackage{tabularx} %Hacer tablas con ancho de columna específico
		\usepackage{enumerate} % Permite usar enumerate
		\usepackage{color} % Permite cambiar color
		\usepackage{cite} % Permite hacer referencias y citas
	
		\usepackage{amsmath} %American Mathematical Society
		\usepackage{mathtools} %Mejora amsmath y latex para usar cosas de matemáticas
		\usepackage{amssymb} % Agrega símbolos útiles como \Cap y \Cup
		\usepackage{amsthm} % Permit usar los comandos teoremas de la AMS
		\usepackage{stmaryrd} % Permite usar dobles paréntesis y corchetes [[ ]]
		\usepackage{centernot, cancel} %Para hacer relaciones y símbolos tachados adecuadamente
		\usepackage{mathrsfs} % Permite usar las fonts \mathscr
	
		%\usepackage{tikz} % Hacer gráficos mediante latex. Hay un libro entero al respecto.
		%\usepackage{diagrams} %Permite hacer diagramas de categorías fácilmente.
		%\usepackage{extarrows} %Permite agregar flechas a los diagramas de categorías.
	
		\usepackage{blindtext} % Para hacer LoremIpsum latinum stuff!!!
		\usepackage{hyperref} % For hyperlinks in the PDF (asegurarse de que está al final)
	
	% Colores
		\definecolor{DB}{RGB}{23,20,119}
		\definecolor{DG}{RGB}{2,101,15}
		\definecolor{DR}{RGB}{120,10,10}
	
	% Theorem Environment
		\newtheoremstyle{JnthnThm}	% name
					{10pt}		% Space above
					{10pt}		% Space below
					{}			% Body font (\itshape es italica para secciones)
					{}			% Indent amount
					{\bfseries\color{DR}}	% Theorem head font (\bfseries es bold para secciones)
					{\\}			% Punctuation after theorem head
					{.5em}		% Space after theorem head
					{}			% Theorem head spec (can be left empty, meaning ‘normal’)
					
		\theoremstyle{JnthnThm}		% Todo los teoremas definidos después tendrán este estilo
			\newtheorem{thm}{Teorema}[section]
			\newtheorem{defn}{Definición}[section]
			\newtheorem{lem}[thm]{Lema}
			\newtheorem{cor}[thm]{Corolario}
			\newtheorem{prprty}[thm]{Propiedades}
			\newtheorem{prop}[thm]{Proposición}
	
		\newtheoremstyle{JnthnExmp}	% name
					{10pt}		% Space above
					{10pt}		% Space below
					{\color{DB}}	% Body font (\itshape es italica para secciones)
					{}			% Indent amount
					{\bfseries\color{DR}}	% Theorem head font (\bfseries es bold para secciones)
					{}			% Punctuation after theorem head
					{.5em}		% Space after theorem head
					{}			% Theorem head spec (can be left empty, meaning ‘normal’)
	
		\theoremstyle{JnthnExmp}		% Todo los teoremas definidos después tendrán este estilo
			\newtheorem{exmp}{Ejemplo}[section]
			\newtheorem{cntrexmp}[exmp]{Contraejemplo}
	
		\makeatletter % Hacer @ leible por latex
		\renewenvironment{proof}[1][\textbf{\proofname}]{\par\pushQED{\qed}
		\normalfont\color{DB} \topsep3\p@\@plus3\p@\relax\trivlist\item\relax
		{\itshape#1\@addpunct{.}}\hspace\labelsep\ignorespaces}{%\itshape (junto a ignoerspaces)
		\popQED\endtrivlist\@endpefalse}
		\makeatother % Regresar @ a su estado normal
	
		\makeatletter % Hacer @ leible por latex
		\newenvironment{sketch}[1][\textbf{Idea de Prueba}]{\par\pushQED{\qed}
		\normalfont\color{DB} \topsep3\p@\@plus3\p@\relax\trivlist\item\relax
		{\itshape#1\@addpunct{.}}\hspace\labelsep\ignorespaces}{%\itshape (junto a ignoerspaces)
		\popQED\endtrivlist\@endpefalse}
		\makeatother % Regresar @ a su estado normal
		
	% New Commands and environments
		\renewcommand{\qedsymbol}{$\blacksquare$}
		\newcommand{\wff}{\emph{wff}} % Usar \wff\ cuando se quiera espacio después de ``wff''
		\newcommand{\wffs}{\emph{wffs}}
		\newcommand{\impl}{\Rightarrow}
		\renewcommand{\iff}{\Leftrightarrow}
		\DeclareMathOperator{\Dom}{Dom}
		\DeclareMathOperator{\Cod}{Cod}
		\DeclareMathOperator{\Ran}{Ran}
		%\newarrow{Equals}=====
	
	% Diseño del encabezado
		\pagestyle{fancyplain}
		\fancyhf{} % delete current setting for header and footer
		\fancyhead[LE,RO]{Jonathan Julián Huerta y Munive}
		\fancyhead[RE,LO]{Notas 0002}
		\fancyfoot[LE,RO]{\thepage}
		\renewcommand{\headrulewidth}{0.5pt}
		\renewcommand{\footrulewidth}{0.4pt}
		\addtolength{\headheight}{0pt} % make space for the rule
		\fancypagestyle{plain}{%
		\fancyhead{} % get rid of headers on plain pages
		\renewcommand{\headrulewidth}{0pt} % and the line
		\cfoot{\thepage} }
		
	\begin{document}
		Make private
		\section{Hybrid Systems Verification with Modal Kleene Algebra}
		% Modal Kleene Algebra for Hybrid Systems Verification
		% Towards the verification of hybrid systems with modal kleene algebra
		\section{Abstract}
		Georg\\

		\section{Introduction}
		Georg\\

		2 pages
		\section{MKA} % (In Isabelle?)
		Georg\\
		1 page\\
		Relation with Dynamic Logic (Axioms $\sim$ Equations)

		\section{(Isabelle Verification Components)}
		Georg\\
		1 page
		\section{Integrating a Hybrid Store} % Differential Dynamic Logic
		\section{Verification Condition Generation}%Deriving rules of a differential dynamic algebra
		\section{Examples}
		% Case Studies
		\section{Conclusion}
		Georg\\
		Half a page (with references: 2 pages)
		\section{Appendices}
		\subsection{Background on Differential Equations}
		\subsection{Axioms of Modal Kleene Algebra}
		Georg\\
		\subsection{Differential Dynamic Logic Rule System (Algebraic version)}
		\subsection{Isabelle Theory Hierarchy}

	\end{document}

%Para hacer relaciones y símbolos tachados adecuadamente
%Para hacer relaciones y símbolos tachados adecuadamente
%Para hacer relaciones y símbolos tachados adecuadamente
%Para hacer relaciones y símbolos tachados adecuadamente
%Para hacer relaciones y símbolos tachados adecuadamente
%Para hacer relaciones y símbolos tachados adecuadamente
%Para hacer relaciones y símbolos tachados adecuadamente
%Para hacer relaciones y símbolos tachados adecuadamente
%Para hacer relaciones y símbolos tachados adecuadamente
%Para hacer relaciones y símbolos tachados adecuadamente
%Para hacer relaciones y símbolos tachados adecuadamente
%Para hacer relaciones y símbolos tachados adecuadamente
%Para hacer relaciones y símbolos tachados adecuadamente
%Para hacer relaciones y símbolos tachados adecuadamente
%Para hacer relaciones y símbolos tachados adecuadamente
%Para hacer relaciones y símbolos tachados adecuadamente
%Para hacer relaciones y símbolos tachados adecuadamente
%Para hacer relaciones y símbolos tachados adecuadamente
%Para hacer relaciones y símbolos tachados adecuadamente
%Para hacer relaciones y símbolos tachados adecuadamente
%Para hacer relaciones y símbolos tachados adecuadamente
%Para hacer relaciones y símbolos tachados adecuadamente
%Para hacer relaciones y símbolos tachados adecuadamente
%Para hacer relaciones y símbolos tachados adecuadamente